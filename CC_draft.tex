\documentclass{article}\usepackage[]{graphicx}\usepackage[]{color}
%% maxwidth is the original width if it is less than linewidth
%% otherwise use linewidth (to make sure the graphics do not exceed the margin)
\makeatletter
\def\maxwidth{ %
  \ifdim\Gin@nat@width>\linewidth
    \linewidth
  \else
    \Gin@nat@width
  \fi
}
\makeatother

\definecolor{fgcolor}{rgb}{0.345, 0.345, 0.345}
\newcommand{\hlnum}[1]{\textcolor[rgb]{0.686,0.059,0.569}{#1}}%
\newcommand{\hlstr}[1]{\textcolor[rgb]{0.192,0.494,0.8}{#1}}%
\newcommand{\hlcom}[1]{\textcolor[rgb]{0.678,0.584,0.686}{\textit{#1}}}%
\newcommand{\hlopt}[1]{\textcolor[rgb]{0,0,0}{#1}}%
\newcommand{\hlstd}[1]{\textcolor[rgb]{0.345,0.345,0.345}{#1}}%
\newcommand{\hlkwa}[1]{\textcolor[rgb]{0.161,0.373,0.58}{\textbf{#1}}}%
\newcommand{\hlkwb}[1]{\textcolor[rgb]{0.69,0.353,0.396}{#1}}%
\newcommand{\hlkwc}[1]{\textcolor[rgb]{0.333,0.667,0.333}{#1}}%
\newcommand{\hlkwd}[1]{\textcolor[rgb]{0.737,0.353,0.396}{\textbf{#1}}}%
\let\hlipl\hlkwb

\usepackage{framed}
\makeatletter
\newenvironment{kframe}{%
 \def\at@end@of@kframe{}%
 \ifinner\ifhmode%
  \def\at@end@of@kframe{\end{minipage}}%
  \begin{minipage}{\columnwidth}%
 \fi\fi%
 \def\FrameCommand##1{\hskip\@totalleftmargin \hskip-\fboxsep
 \colorbox{shadecolor}{##1}\hskip-\fboxsep
     % There is no \\@totalrightmargin, so:
     \hskip-\linewidth \hskip-\@totalleftmargin \hskip\columnwidth}%
 \MakeFramed {\advance\hsize-\width
   \@totalleftmargin\z@ \linewidth\hsize
   \@setminipage}}%
 {\par\unskip\endMakeFramed%
 \at@end@of@kframe}
\makeatother

\definecolor{shadecolor}{rgb}{.97, .97, .97}
\definecolor{messagecolor}{rgb}{0, 0, 0}
\definecolor{warningcolor}{rgb}{1, 0, 1}
\definecolor{errorcolor}{rgb}{1, 0, 0}
\newenvironment{knitrout}{}{} % an empty environment to be redefined in TeX

\usepackage{alltt}
\usepackage{graphicx}
\graphicspath{{images/}}
\IfFileExists{upquote.sty}{\usepackage{upquote}}{}
\begin{document}

\textbf{Background: Footwear and Forensic Analysis}

In forensic science, shoe prints and outsole characteristics fall into the category of pattern evidence. When a shoe print is found at a crime scene, there are a number of questions that can be asked. For example, an investigator may want to determine the make or model of the show that made the print, or possibly tie the shoe print to a specific individual. Another question, more difficult than the first two, is how common the shoe type, or features of that shoe type, are in a local population. This question is more difficult than the first two because it relies not only on the information contained in the found print, but also on characteristics of many other types other shoes. Thus, any sufficient answer to this question requires a way to automatically and efficiently classify many different types of shoes within a common system.


\textbf{Classification}

Visual classification is a complex task that our brains have been trained to do very well. Our eyes detect a large variety of features of an object, including color, shape, and texture, and send that information to our brain. Our brain then learns which combinations of features to associate with a given label, and should be able to apply those rules to future objects with similar characteristics. For example, an orange caterpillar and a baby carrot may be of similar color, shape, and size, but one is distinctly more fuzzy than the other. Thus, our brains learn that when faced with a small, cylindrical orange object, texture becomes an important feature when assigning a label to that object (which keeps us from accidentally ingesting caterpillars). 

\textbf{Convolutional Neural Networks (CNNs)}

While our brains are adept at parsing images and classifying the objects within them, the task has proved much more difficult for computers. Convolutional neural networks (CNNs) are a tool for supervised deep-learning that have become standard in recent years for automatic image classification. CNNS use combinations of convolutional and pooling hidden layers to filter raw information into features, which are then fed into densely connected layers which are trained to associate given sets of features with their desired labels. This translation-invariant automated classification mimics the human eye-to-brain classification process and has become one of the most widely used machine learning techniques for image classification.


\textbf{Pre-trained CNNs}

%The structure of convolutional neural networks can vary depending on the type of classification the model is designed for, as well as the desired speed and accuracy. While networks can be built and trained from scratch, to do so typically requires a large amount of computing power and millions of images for any practical classification task. Another common way to use CNNS for a novel task is to use a pre-trained network 

*I don't love this paragraph. Not sure how much to include, awkward flow.*
Pre-trained CNNs are CNNs that have been trained on a standard data set. The standard data set comes from ImageNet, a database containing over 14 million images in about 22,000 categories (called "synsets", short for "synonym sets"). The ImageNet Large Scale Visual Recognition Challenge (ILSVRC) was established in 2010 as a contest for CNN accuracy on a specific subset of ImageNet. Various CNN structures are tested on about 1.2 million images spanning 1,00 categories. These categories range from natural and man-made objects (e.g., daisy, chainsaw) to living creatures (e.g., ring-tailed lemur, sea lion, and dingo). There are also many categories which require subtle distinctions, such as differentiating between a grass snake and a vine snake. *Something about the best CNNs for this task are the ones that are famous.* Some of the most well-known pre-trained CNNs include AlexNet, GoogLeNet/Inception, VGG, and ResNet. *Can use just structure and train weights yourself, use fully trained model to reproduce ILSVRC results, or just use pre-trained weights for feature detection*

\textbf{VGG16 Architecture}
The main difference between different CNNs is their structure, meaning the number of layers they contain and the pattern those layers are in. In our research, we have tested a few pre-trained CNNs, and we are currently using VGG16. Developed by Oxford's Visual Graphics Group, VGG16 has 16 "functional" (i.e., convolutional and densely connected) layers and 5 max-pooling layers, which function more to alter the structure of the information at each step.

\textbf{Filters and Convolution}
Convolutional Neural Networks are named to highlight their use of convolution to extract information from an image. To a computer, an image is stored as a 3-dimensional array with a length and width corresponding to its number of pixels and a depth of 3 to represent the typical RBG color channels. A single convolutional filter is a small array (say 5x5x3) of real valued weights that represents some feature of the image. When a filter is applied to a portion of the image, the weights are multiplied with the image values and all values are summed, which returns a single value associated with how strong the presence of the feature is for that part of the image. When applied over an entire image, the resulting matrix of values maps the strength of the feature across the entire image. A convolutional layer of a CNN takes a large number of these filters and passes them over the image to return one feature map per filter. 

In the case of VGG16, early convolutional layers contain 64 features that primarily detect colors and edge patterns. Later convolutional layers of VGG16, in contrast, contain 512 filters that represent much more complex features, like animal fur patterns or distinct bird heads.

\textbf{Max-Pooling}
Max-pooling is a technique to reduce the size, and therefore computational load, of feature maps through structured down-sampling. Max-pooling layers apply a maximum function over adjacent regions of a feature map (like using a sliding window) to encode the important information of how strongly a feature was activated in a given region of the image while simultaneously reducing redundant or unneccessary information about smaller activations. For example, taking 2x2 pieces of a feature map and keeping only the largest of the four values reduces the size of the feature map by a factor of 4! Max-pooling is also beneficial in that it allows CNN "vision" to be translation invariant, because it emphasizes the relative position of a feature rather than its absolute position. VGG16 follows groups of 2 or 3 convolutional layers with a max-pooling layer, which ultimately takes in initial feature maps of size 224x224 and ends with maps of size 7x7.

\textbf{Densely Connected Layers}
Densely connected layers are typically the final layers in a CNN. These layers form the meaningful connection between the features of an image (detected by convolutional and max-pooling layers) and the corresponding labels associated with the image. These layers act like the human brain: just as we learned which combinations of features should be associated with a given label, densely connected layers use real-valued weights to represent these associations. For example, if we see an item that is orange, small, and fuzzy, we are taught to call it "caterpillar". Fuzzzy is not a feature we meaningfully associate with a baby carrot, so there is little connection between the feature "fuzzy" and the label "carrot". Similarly, in CNNs, each final feature is connected to each label through a weight (hence the name "densely connected"), and those weights are learned through the training process (using an algorithm called back-propogation) to minimize loss and thus improve classification accuracy.


\textbf{Using a Pre-Trained CNN for a New Task}
*I went off on a tangent and am purposely not restructuring this paragraph yet. Sorry.*
As we have just seen, convolutional layers and max-pooling layers in a CNN are analogous to the human visual perception process, and densely connected layers behave like the human brain. In short, the approach to classifying an image is to detect the features in the image (like our eyes) and then assign labels to combinations of those features (brain). This analogy is also appropriate because it reflects the difficulty of the task: it takes many years and a significant amount of effort for humans to learn how to distinguish a large variety of features and also to connect those features to labels that are often complex, hierarchical, and subtle. Similarly, training a CNN is no small task. VGG16, in particular, has over 14.7 million trainable parameters in its "eyes" alone. Luckily, CNNs offer one benefit that humans do not: you can utilize the eyes and replace the brain for new tasks. In terms of CNNs, it is possible to build a CNN that uses the weights already trained on over 1.2 million images in the convolutional layers, and then only retrain a new classifier for any new classification task. This reduced task brings the number of required training images down from millions to only thousands. Furthermore, this kind of approach is quite reasonable when considering what the CNN was originally trained to classsify. Since the 1,000 categories from the ILSVRC span a huge variety of natural and unnatural objects, we can likely trust that the features detected by the pre-trained CNN to be diverse enough to be applied to a new task.

\textbf{Outsole Class Characteristics}

*Transition from CNN to shoes* In the beginning of this project, the literature (source IDK) indicated that geometric shapes are unique and well-defined enough to provide most of the necessary information to classify a shoe outsole. 

\end{document}
