\specialchapt{ABSTRACT}

A common goal of forensic shoeprint analysis is to identify shoe models or designs that are similar to that of a given print, such as a print found at a crime scene. Quantifying similarity between shoe outsole patterns is difficult because it requires both a set of well-defined features and an accurate method to classify outsoles according to those features. A set of geometric features was developed based on common geometric shapes, such as circles and quadrilaterals. A new classifier was then trained for the pretrained convolutional neural network base of VGG16 to create a model, named CoNNOR, to automatically classify portions of outsole images into the new geometric scheme of outsole class characteristics. During the analysis of CoNNOR's performance, new diagnostic plots were developed which provide a better method to assess classification errors of multi-class, multi-label models. In general, CoNNOR performs well on images with unambiguous shapes and moderate color contrast; additional improvements may be realized by preprocessing the images to improve contrast as well as by integrating spatial relationships between geometric features. CoNNOR represents a significant improvement to the current manual classification of footwear tread patterns, facilitating new modes of data collection and automatic processing that will expand the data available for assessment of footwear class characteristic frequency in the population.
