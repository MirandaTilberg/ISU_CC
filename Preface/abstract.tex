\specialchapt{ABSTRACT}

A common goal of forensic shoe print analysis is to identify shoe models or designs that are similar to that of a given print, such as a print found at a crime scene. Quantifying similarity between shoe outsole patterns is difficult because it requires both a set of well-defined features and an accurate method to classify outsoles according to those features. A set of geometric features was developed based on common geometric shapes, such as circles and quadrilaterals, that is broad enough to encompass a large variety of designs and \mt{specific enough to differentiate between designs with similar ?}. A new classifier was then trained for the pre-trained convolutional neural network base of VGG16 to create a model, named CoNNOR, to classify portions of outsole images into the geometric scheme of outsole class characteristics. In general, CoNNOR performs well on images with unambiguous shapes and favorable color contrast, but there is substantial evidence that contrast correction may improve overall model performance. Future work may also seek to alter the convolutional base to bypass layers that are overly complex for the simple geometric features of outsoles, and then to explore spatial integration of predictions to map features for entire outsoles.
