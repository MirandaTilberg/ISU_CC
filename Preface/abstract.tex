\specialchapt{ABSTRACT}

A common goal of forensic shoe print analysis is to identify shoe models or designs that are similar to that of a given print, such as a print found at a crime scene. Quantifying similarity between shoe outsole patterns is difficult because it requires both a set of well-defined features and an accurate method to classify outsoles according to those features. \svp{Using} a set of geometric features based on common geometric shapes, such as circles and quadrilaterals\svp{, a} new classifier\svp{, CoNNOR was developed using transfer learning}. \svp{CoNNOR automatically classifies images according to the basis set of geometric features, leveraging the pre-trained neural network VGG16 as well as a set of labeled shoe treads to automatically identify relevant class characteristics}. \svp{During the analysis of CoNNOR's performance, new diagnostic plots for this type of model were developed which provide a better method to assess classification errors in multi-class, multi-label models}. In general, CoNNOR performs well on images with unambiguous shapes and \svp{moderate} color contrast; \svp{additional improvements may be realized by preprocessing the images to improve contrast as well as by integrating spatial relationships between geometric features}. \svp{CoNNOR represents a significant improvement to the current manual classification of footwear tread patterns, facilitating new modes of data collection and automatic processing that will expand the data available for assessment of footwear class characteristic frequency in the population.} %Future work may also seek to alter the convolutional base to bypass layers that are overly complex for the simple geometric features of outsoles, and then to explore spatial integration of predictions to map features for entire outsoles.
